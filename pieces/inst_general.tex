% inst_general.tex
%--------------------------------------------------------------------
% For writing instructions
%--------------------------------------------------------------------
\newcommand{\KLInstWOGeneral}[1]{%
	\noindent
 ーー ※留意事項 ーーーーーーーーーーーーーーーーーーーーーーーーーーーーーーーーーー\\
		#1\\
 ーーーーーーーーーーーーーーーーーーーーーーーーーーーーーーーーーーーーーーーーーー
}

\newcommand{\KLInst}[1]{%
	\noindent
	\ifthenelse{\equal{#1}{}}{%
 ーー ※留意事項 ーーーーーーーーーーーーーーーーーーーーーーーーーーーーーーーーーー\\
	}{%
 ーー ※留意事項\textcircled{1} ーーーーーーーーーーーーーーーーーーーーーーーーーーーーーーーーー\\
		#1\\
		
	\noindent
 ーー ※留意事項\textcircled{2} ーーーーーーーーーーーーーーーーーーーーーーーーーーーーーーーーー\\
	}
}

\newcommand{\GeneralInstructions}{%
  1.作成に当たっては、研究計画調書作成・記入要領を必ず確認すること。\\
  2.本文全体は11ポイント以上の大きさの文字等を使用すること。\\
  3.各頁の上部のタイトルと指示書きは動かさないこと。\\
  4. 指示書きで定められた頁数は超えないこと。なお、空白の頁が生じても削除しないこと。\\
  \textcolor{red}{5.本留意事項は、研究計画調書の作成時には削除すること。(\texttt{\textbackslash JSPSInstructions}を消す)}\\
 ーーーーーーーーーーーーーーーーーーーーーーーーーーーーーーーーーーーーーーーーーー
}
